\chapter{Algorithme de Barnes-Hut}

Pour accélérer les calculs et permettre de plus grandes simulations, nous allons dans cette partie nous intéresser à l’algorithme de Barnes-Hut.

\section{Présentation de l'algorithme}

L'algorithme de Barnes-Hut est un algorithme hiérarchique (cf. un algorithme qui classe les calculs à effectuer en fonction d'un paramètre) inventé par Josh Barnes et Piet Hut en 1986. Il est basé sur l'utilisation d'un arbre appelé $quadtree$ afin d'approximer le calcul des  interactions gravitationnelles. Il permet de réduire les calculs de manière à obtenir une complexité en $O(Nlog(N))$, tout en restant physiquement correct. Sa fiabilité et son efficacité en fait alors l'algorithme le plus utilisé pour résoudre le problème à N-corps.
 
\section{Principe général de l'algorithme}

L'idée est d'approcher les forces à longue portée en considérant un groupe de points éloignés comme équivalent à leur centre de masse. Il y a évidemment en contrepartie une légère erreur due à l'approximation mais ce schéma accélère considérablement le calcul. Cette erreur peut être contrôlée à partir d'un paramètre appelé $\theta$. Cet algorithme a une complexité en $O(Nlog(N))$ au lieu du $O(N^2)$ des méthodes naïves.
Au centre de cette approximation se trouve un arbre : une « carte » de l'espace qui nous aide à modéliser des groupes de points comme un seul centre de masse. En deux dimensions, nous pouvons utiliser une structure de données $\textit{quadtree}$, qui subdivise de manière récursive les régions carrées de l'espace en quatre quadrants de taille égale. En trois dimensions, on peut utiliser un octree qui divise de la même manière un cube en huit sous-cubes.


\begin{center}
\includegraphics[scale=0.2]{./images/quadtree.png}
\captionsetup{hypcap=false}
\captionof{figure}{Exemple de Quadtree \\ \url{https://www.lrde.epita.fr}}
\label{fig6}
\end{center}

\section{L'algorithme de Barnes-Hut}
L'algorithme de Barnes Hut se décompose en trois étapes majeures:

\begin{enumerate}
\item la construction de l'arbre

\item le calcul des centres des masses et des masses correspondants à chaque nœud

\item le calcul des forces appliquées sur les particules
\end{enumerate}

Selon la version de l'algorithme utilisée, les étapes $1$ et $2$ peuvent être effectuées simultanément. 

Avant de détailler ces étapes nous avons besoin de mettre en place la notion de quadrant. Un quadrant est une division de notre arbre et donc un nœud de l'arbre. Comme nous travaillons en deux dimensions, nous avons besoin de quatre quadrants, un quadrant en haut à gauche, un autre en bas à gauche , un troisième à droite en haut et un dernier à droite en bas. On les appellera respectivement NW (North-West),SW (South-West),NE (North-East) et SE (South-East) en référence aux points cardinaux.

\subsection{La construction de l'arbre}

Elle consiste à insérer successivement les particules dans l'arbre.

Soit la particule $p1$. Tout d'abord nous devons vérifier que $p1$ doit bien être insérée dans l'arbre.
Si c'est le cas nous distinguons plusieurs cas. 
\begin{itemize}

\item S'il existe plus d'une particule dans le nœud, on doit insérer la particule dans le fils du nœud auquel correspond sa position donc dans un des quatre quadrants. S'il n'existe pas, on crée ce fils.


\item S'il existe qu'une seule particule dans le nœud, on réinsère l'ancienne particule dans le fils auquel correspond sa position puis on effectue l'insertion de $p1$ dans le quadrant qui lui correspond. Si les fils n'existent pas, on les crée avant l'insertion. 

\item Enfin s'il n'y a aucune particule dans le nœud, il s'agit  d'une feuille, on insère directement $p1$ sans créer de fils.

\end{itemize}

Les particules ne sont insérées que dans les feuilles de l'arbre, les nœuds peuvent contenir plusieurs particules mais les particules en elles-mêmes ne seront stockées dans l'arbre que dans des feuilles de l'arbre (nœuds sans fils). Ainsi, on découpe notre à arbre en $4$ à chaque itération jusqu'à avoir inséré toutes les particules et que chaque carré ne contienne qu'au plus $1$ particule comme on peut le voir dans la figure 5.1.

\subsection{Le calcul des masses et centres de masse}

Les calculs des masses et centres de masses se calculent naturellement en utilisant la récursivité en commençant par les feuilles de l'arbre.

\vspace{2mm}
Soit le nœud $n$. 
Si $n$ ne contient qu'une seule particule alors il s'agit d'une feuille et sa masse correspond à la masse de la particule et son centre de masse sera la position de la particule. Si ce n'est pas le cas, la masse de $n$ sera la somme des masses de chacun des fils et son centre du masse se calcule par la formule suivante

\begin{equation}
    CM_n = \frac{m_1*CM_1 + m_2*CM_2+ m_3*CM_3 + m_4*CM_4 }{\sum_{i=1}^{4}{m_i}}
\end{equation}

où $\forall i\in \{1,2,3,4\}$, $CM_i$ est le centre de masse d'un fils et $m_i$ est sa masse.

\subsection{Le calcul des forces}

L'avantage de l'utilisation d'un arbre est que le calcul des forces gravitationnelles pour chaque particule se fait simplement en se servant de la récursivité, à partir de la racine de l'arbre. De plus, comme dit précédemment, le niveau de précision de l'algorithme et donc le nombre de calculs dépendent du paramètre $\theta$ qu'on fixe autour de $1$. Nous prendrons $\theta = 0.9$.

Pour chaque particule $p$, on parcourt l'arbre depuis sa racine.

\begin{itemize}
\item Si le nœud actuel est une feuille, on calcule simplement la force exercée par la particule qu'elle contient sur la particule actuelle.

\item Sinon, on calcule le rapport $\frac{l}{D}$ où $l$ est la
longueur d'un coté du nœud actuel  et $D$ est la distance entre la particule $p$ et le centre de masse du nœud actuel.
Si $\frac{l}{D} < \theta$ alors on calcule la force, appliquée à $p$ à partir du centre de masse et de la masse du nœud. Sinon, elle sera la somme des forces exercées par les fils de ce nœud.
\end{itemize}

\newpage
Nous avons ainsi par exemple le résultat suivant:
\begin{center}
\includegraphics[scale=0.8]{./images/BH_tree.png}
\captionsetup{hypcap=false}
\captionof{figure}{Arbre construit avec l'algorithme de Barnes Hut avec $20000$ particules}
\label{fig7}
\end{center}

Il devient alors possible de pousser les simulations plus loin pour simuler notamment la formation de galaxies (spirales ou sphériques) en augmentant considérablement le nombre de particules. Cela montre alors l'intérêt et l'efficacité de l'algorithme de Barnes-Hut.

\section{Vérification du modèle}

\begin{center}
\includegraphics[scale=0.6]{./resultats/Energy_BH.png}
\captionsetup{hypcap=false}
\captionof{figure}{Bilan énergétique méthode de Barnes-Hut \\($\Delta t = 50$ ans, $1000$ itérations et $2000$ particules)}
\label{fig9}
\end{center}

L'énergie mécanique se conserve à partir de la $260$ème itération, avant ça, le modèle est perturbé par les conditions initiales. Ainsi, le modèle est bien valide physiquement malgré les approximations effectuées, ce qui montre la puissance de l'algorithme. Il a une vitesse de convergence située entre celles des deux méthodes naïves.
