\chapter{Conclusion}

Le projet consistait à étudier différentes manières de résoudre numériquement le problème à N-corps afin d'obtenir la solution la plus efficace.

Nous avons pu mettre en place $3$ méthodes de calcul différentes pour les forces :

\begin{itemize}
\item la méthode naïve qui calcule pour chaque particule les forces qui s'y applique en répétant les calculs

\item la méthode naïve optimisée qui utilise le principe d'action-réaction afin de diminuer le nombre de calculs

\item l'algorithme de Barnes-Hut qui utilise un arbre afin réduire considérablement le nombre de calcul en utilisant des approximations.

\end{itemize}

Au final, l'algorithme de Barnes-hut est bien plus rapide et efficace que les autres malgré qu'il donne des résultats similaires. Il s'agit donc de la meilleure résolution approchée possible au problème à N-corps. De plus, la parallélisation permet d'avoir des performances plus stables et montre son intérêt pour de grandes simulations. Cependant, la parallélisation des autres méthodes montre clairement l'importance d'avoir une implémentation qui soit compatible, au risque d'avoir une parallélisation inefficace ou même pénalisante.

Cependant, ce projet reste encore exploitable puisqu'il existe encore beaucoup de possibilités comme par exemple le passage sur une carte graphique pour simuler plus de points ou encore le passage en 3 dimensions.


