\chapter{Conclusion}

\par Nous avons étudié un code qui nous a été fournit afin de commencer une implémentation naïve du problème à N-corps. Avant de résoudre naïvement le problème, nous avions recrée les fonctions nécessaires à l'initialisation  à la simulation (nombre de trous noirs galactiques, taille des galaxies ou encore autre forme de départ plus singulière).
\par Dans un deuxième temps, nous sommes allés un peu plus loin et nous nous sommes servies de l'algorithme de Barnes-Hut afin d'améliorer l'efficacité de notre simulation. Enfin nous nous sommes servies de la bibliothèque OpenMP pour faire du multi-thread et ainsi rendre la simulation beaucoup plus efficace.
\par Comme vous l'avez pu constater à travers notre rapport, l’objectif du projet a bien été réussi puisque nous avions réussi à simuler le plus efficacement possible à notre échelle le problème à N-Corps.
\par Ce projet reste encore exploitable puisqu’il existe encore beaucoup de possibilité comme par exemple le passage sur une carte graphique pour simuler plus de point ou encore le passage à la 3 dimensions.
