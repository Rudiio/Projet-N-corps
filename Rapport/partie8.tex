\chapter{Conclusion du projet}

\section{Conclusion}

Le projet consistait à étudier différentes manières de résoudre numériquement le problème à N-corps afin d'obtenir la simulation la plus satisfaisante. Au final, nous avons implémenté $3$ méthodes de calcul différentes pour les forces :

\begin{itemize}
\item la méthode naïve qui calcule pour chaque particule les forces qui s'y applique de la manière la plus basique possible

\item la méthode naïve optimisée qui utilise le principe d'action-réaction afin de diminuer le nombre de calculs

\item l'algorithme de Barnes-Hut qui utilise un arbre afin de réduire considérablement le nombre de calculs en utilisant des approximations.

\end{itemize}

\vspace{2mm}
Au final, l'algorithme de Barnes-hut est bien plus rapide et efficace que les autres, et malgré les approximations qu'il fait, il produit une évolution valide. Il constitue pour le moment la meilleure résolution approchée possible au problème à N-corps. De plus, la parallélisation permet d'avoir des performances bien plus intéressantes et montre notamment son intérêt lorsqu'on augmente le nombre de particules à plusieurs dizaines de milliers. Cependant, la parallélisation de l'insertion des particules dans l'arbre montre l'importance de la compatibilité des algorithmes, au risque d'avoir une parallélisation inefficace, voire même une inconsistance de nos calculs.

\vspace{2mm}
De nombreux axes de développement du projet sont possibles tel que l'optimisation de la parallélisation, l'utilisation d'une carte graphique pour simuler encore plus de particules, ou l'extension de la simulation en 3D, afin d'observer des structures qui ne sont pas nécessairement stables en 2D.

\section{Impacts du projet}

La résolution du problème à N-corps est extrêmement importante dans des domaines tels que l'astronomie. Elle permet aux scientifiques de ces domaines d'émettre des hypothèses ou de les vérifier de façon accessible et rapide, en modifiant les paramètres au besoin, du fait que les objets d'études des disciplines liées sont difficilement observables et manipulables.

\vspace{2mm}
De plus, l'optimisation du calcul a un fort impact sur la recherche dans ces domaines, le calcul haute-performance étant un nouvel enjeu de nos sociétés. En effet, la réduction des temps de calculs permet dans un premier temps d'augmenter les volumes de données traitées, ainsi que la précision des simulations. De plus, réduire les temps de calcul permet de réduire l'énergie dépensée, et donc de diminuer la pollution liée à la production énergétique. Ces améliorations sont indispensables à de plus grandes échelles. 

\section{Bilans personnels}

\subsection{Camille}

Je pense que ce projet m’a beaucoup aidé en informatique,
notamment par l'apprentissage d’un tout nouveau langage (C++).
J’ai beaucoup aimé le principe de l’algorithme de Barnes-Hut et
avoir travaillé et réfléchi dessus a été très formateur. Je pense
avoir apporté des idées au groupe pour nos algorithmes, ainsi que
des points de vue différents pour d’autres approches et une
meilleure compréhension globale.

\subsection{Noah}

Ce projet, à l'intersection de la physique, des mathématiques et de l'informatique, nous a permis de développer nos compétences techniques en C++, parallélisation et optimisation. L'interdisciplinarité m'a vraiment plu car cela nous a permis de nous intéresser à beaucoup de chose différentes. Cependant, le fait que la majeure partie du code était déjà fournie nous a contraint à étudier le code pendant une longue période et nous a confronté à des difficultés d'implémentation de certaines fonctions. Je pense avoir su contribuer au développement du projet en suggérant des idées adaptées aux difficultés rencontrées.

\subsection{Rudio}

Au final, à travers ce projet, j'ai pu développer des connaissances dans différents domaines. Il m'a poussé à dépasser mes limites pour comprendre le fonctionnement du code qui était dans un langage que je ne maîtrisais pas et pour assimiler les aspects physiques et mathématiques du problème.
Cela m'a donc permis de connaître les principaux aspects et concepts du C++ et de la programmation orientée objet mais aussi de la parallélisation multi-thread avec OpenMP. J'ai également pu améliorer mes connaissances en mécanique et en résolution numérique. 
Ainsi, malgré les difficultés rencontrées, j'ai vraiment apprécié le projet et son déroulement notamment grâce aux résultats que nous avons pu obtenir. De plus, le travail de groupe m'a plu étant donné que chaque membre du groupe s'est investi et a apprécié le projet.
J'aurais cependant aimé pouvoir poussé le projet plus loin en explorant d'autres façons d'optimiser le programme.


\subsection{Elyas}

Le projet était exactement ce que j'espérais, le fait que nous avons procédé à la simulation d'objets célestes m'a permis d'apprendre plusieurs notions techniques (au niveau informatique) mais aussi scientifiques. L'utilisation du C++ a été aussi très intéressante cependant je regrette qu'on n'ait pas exploité plus ce langage. Ce projet m'a aussi permis de voir que l'utilisation des mathématiques est un outil très important notamment pour l'amélioration des calculs avec la mise en place des différents intégrateurs.


