\chapter{Conclusion}

\par Jusque-là, nous avons étudié le code qui nous a été fournit afin de commencer une implémentation naïve du problème à N-corps. Avant de résoudre naïvement le problème, nous allons recrée les fonctions nécessaires à l'initialisation  à la simulation (nombre de trous noirs galactiques, taille des galaxies ou encore autre forme de départ plus singulière).
\par Nous reimplémenterons par la suite un intégrateur saute-mouton afin de remplacer celui qui était présent dans le code et qui ne fonctionne pas puis nous coderons tout ce qui permettra à la simulation naive de tourner. 
\par Dans un deuxième temps, nous allons nous servir de l'algorithme de Barnes-Hut afin d'améliorer l'efficacité. Enfin nous allons nous servir de la bibliothèque OpenMP pour faire du multi-thread et ainsi rendre la simulation beaucoup plus efficace.
